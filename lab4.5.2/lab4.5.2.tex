\documentclass{MagicLabs}

\begin{document}	
	
\begin{minipage}[l]{0.3\textwidth}
	\textit{Работу выполнил}\\
	Просвирин Кирилл, 712гр.\\\\
	\textit{под руководством}\\
	А.А. Казимирова
\end{minipage}
\hfill
\begin{minipage}[l]{0.21\textwidth}
	Маршрут \RomanNumeralCaps{9} \\\\
	2 марта 2019~г.,\\
	9 марта 2019~г.\\
\end{minipage}
\\[20pt]
\begin{center}
	\LARGE{\textbf{Лабораторная работа № 4.5.2}\\
		Интерференция лазерного излучения\\[20pt]}
\end{center}



\textbf{Цель работы: }исследовать зависимость видности интерференционной картины от разности хода интерферирующих лучей и от их поляризации.

\textbf{В работе используется: }Не-Nе лазер, интерферометр Майкельсона с подвижным зеркалом, фотодиод с усилителем, осциллограф С1-76, поляроид, линейка.


\section{Теоретическая справка}

\subsection*{Гелий-неоновый лазер}

\begin{wrapfigure}[6]{r}{.35\textwidth}\centering
	\vspace{-3ex}
	\includegraphics[width=.35\textwidth]{HeNeLazer}
	\caption{Схема лазера}
	\label{Lazer}
\end{wrapfigure}

Схема лазера приведена на рис. \ref{Lazer}. Газоразрядная трубка Т наполнена
смесью гелия и неона. Торцы трубки  закрыты плоскопараллельными стеклянными или
кварцевыми пластинками $ П $ и $ П' $, установленными под углом Брюстера к оси трубки.
Вследствие этого лазер генерирует 
линейно поляризованное излучение. Для излучения, распространяющегося вдоль оси
интерферометра, наступает резонанс, если на длине интерферометра $ L $ 
укладывается целое число $ m $ полуволн световых колебаний $ L = m\lambda_m/2 $, 
что соответствует частотам
\begin{equation}\label{frequency}
	f_m=\dfrac{c}{\lambda_m}=\dfrac{mc}{2L},
\end{equation}
 
 где $ L $~– длина резонатора, $ m $ – целое число. Тогда можно сформулировать 
 условие на разность частот излучения
 \begin{equation}\label{kek}
 	\nu = f_{m+1} - f_m = \dfrac{c}{2L}.
 \end{equation}

Таким образом, лазер будет генерировать сразу несколько световых волн с различными
частотами. Каждую такую волну называют модой.

Также стоит отметить, что вследствие тепловых флуктуаций длина резонатора меняется, в результате чего моды <<переползают>> с одного края контура на другой, там исчезают, а на другом краю рождаются новые. Это приводит к медленным изменениям амплитуд колебаний в лазерных модах и числа самих мод.

\subsection*{Видность интерференционной картины.}

Если в плоскости наблюдения сходятся под малым углом $ \varphi $две плоских волны 
с длиной волны $ \lambda_0 $, то наблюдается интерференционная картина в виде
 последовательности тёмных и светлых полос с расстоянием между полосами
 \begin{equation}\label{delta_x}
 	\Delta x = \dfrac{\lambda_0}{\varphi}.
 \end{equation}

Для описания чёткости интерференционной картины в некоторой точке введен
параметр видности $ \gamma $:
\begin{equation}\label{visibility_definition}
	\gamma = \dfrac{I_{max} - I_{min}}{I_{max} + I_{min}},
\end{equation}

где $ I_{max}  $ и $ I_{min} $~---~максимальная и минимальная интенсивности света
интерференционной картины вблизи выбранной точки. Параметр $ \gamma $ 
меняется в пределах от 0 (полное исчезновение интерференционной картины) до 1 (наиболее чёткая картина).

\textit{Видность }интерференционной картины  зависит от:
\begin{enumerate}
	\item \textbf{Отношения амплитуд интерферирующих волн.}
	Пусть в плоскости наблюдения интерферируют 
	две волны с амплитудами $ A_m $ и $ B_m $. Тогда интенсивность света в этой точке
	\begin{equation}\label{intensity}
	I_m = A^2_m + B^2_m + 2A_mB_m\cos{k_ml}.
	\end{equation}
	
	Замечаем, что $ I_{max} = (A_m + B_m)^2 $, а в минимуме 
	$ I_{min} = (A_m - B_m)^2 $. Тогда, вводя параметр $  \delta = B_m^2/A_m^2 $, находим
	\begin{equation}\label{visibility1}
		\gamma_1 = \dfrac{2\sqrt{\delta}}{1 + \delta},
	\end{equation}	
	
	\item \textbf{Cпектрального состава света и геометрической разности хода.} 
	Здесь без вывода примем на веру, что функция,
	которая описывает эту зависимость имеет вид
	\begin{equation}\label{visibility2}
		\gamma_2(l) = \dfrac{\displaystyle \sum_{n=1}A_n^2\cos{\left(\dfrac{2\pi\Delta\nu nl}{c}\right)}}
										{\displaystyle \sum_{n=1}A_n^2},
	\end{equation}
	
	где $ l $~---~разность хода интерферирующих лучей, $ A_n^2 $~---~интенсивность мод.
	
	\item \textbf{Поляризации}. Если обе волны линейно поляризованы, а угол между 
	плоскостями их поляризации равен $ \beta $, то в последнем члене формулы \ref{intensity} появится сомножитель~$ \cos\beta $
	\begin{equation}\label{visibility3}
		\gamma_3 = |\cos\beta|.
	\end{equation}
\end{enumerate}


Из вышесказанного следует, что полная зависимость видности от угла между плоскостями поляризации интерферирующих волн, отношения их интенсивностей и разности хода определяется выражением
\begin{equation}\label{visibility}
	\gamma = \gamma_1\gamma_2\gamma_3.
\end{equation}
\newpage
\section{Экспериментальна установка}

\begin{wrapfigure}[17]{r}{.6\textwidth}\centering
	\vspace{-3ex}
	\includegraphics[width=.6\textwidth]{apparatus}
	\caption{Схема установки}
	\label{apparatus}
\end{wrapfigure}

\ \ \ Экспериментальная установка представляет собой интерферометр Майкельсона,
смонтированный на вертикально стоящей плите. (рис. \ref{apparatus}) Источником света служит 
гелий-неоновый лазер ($ \lambda_0 = 632,8$~нм). Луч лазера отражается от зеркала З 
и проходит призму полного внутреннего отражения ПФ (параллелепипед Френеля), 
на выходе из которой он имеет поляризацию, близкую к круговой.
Далее луч света делится диагональной плоскостью делительной призмы ПД на два луча.
Интенсивность света регистрируется фотодиодом Ф, свет на которой попадает через 
узкую щель в центре экрана. Щель ориентируется параллельно интерференционным полосам. 

\begin{wrapfigure}[10]{l}{.4\textwidth}\centering
	\vspace{-3ex}
	\includegraphics[width=.35\textwidth]{photodiod_signal}
	\caption{Осциллограмма сигналов фотодиода}
	\label{PhD}
\end{wrapfigure}

Осциллограммы сигналов фотодиода приведены на рис \ref{PhD}. Осциллограф используется для
регистрации следующих сигналов: фоновой засветки (линия 0 — перекрыты оба луча 1 и 2);
интенсивности света одного из пучков (линии 1 или 2 — перекрыт луч 2 или 1); максимума 
и минимума интенсивности интерференционной картины (открыты оба луча). 
При этом параметр $ \delta $, необходимый для расчёта $ \gamma_1 $ в формуле (7),
определяется отношением
\begin{equation}\label{delta}
	\delta = \dfrac{h_1}{h_2}~~~\left(или~\dfrac{h_2}{h_1}\right)
\end{equation}

Видность интерференционной картины рассчитывается по формуле:
\begin{equation}\label{gamma}
	\gamma = \dfrac{h_4 - h_3}{h_4 + h_3}.
\end{equation}

Измерив величины $ h_1, h_2, h_3 $ и $ h_4, $ можно рассчитать $ \gamma $ и $ \gamma_1 $, 
а затем определить видность при данной разности хода $ l $ для угла между плоскостями
поляризации лучей $ \beta = 0~(\gamma_3 = 1): $
\begin{equation}\label{gamma1}
	\gamma_1(l) = \dfrac{\gamma}{\gamma_1};
\end{equation}

или при $ l = 0,~(\gamma_2 = 1) $ для известного угла $ \beta $:
\begin{equation}\label{gamma3}
	\gamma_3(|\cos\beta|) = \dfrac{\gamma}{\gamma_1}.
\end{equation}

\section{Измерения}

\begin{enumerate}
	\setcounter{enumi}{0}
	
	\item Исследуем зависимость видности интерференционной картины от угла $ \beta $
	поворота поляроида П1 при нулевой разности хода ($ \gamma_2 $ = 1). Для этого измерим
	величины $ h_1, h_2, h_3 $ и $ h_4, $на экране осциллографа. Результаты измерений занесем
	в таблицу \ref{table1}.
	
	Используя формулы \eqref{visibility1}, \eqref{delta}, \eqref{gamma}, \eqref{gamma3} 
	рассчитаем коэффициент $ \gamma_3 $.

\begin{table}[!h]\centering
	\begin{tabular}{cc|cccc|cccc|c} \toprule
		$ \beta, ^\circ $ & $ \beta $, рад & $ h_1 $ & $ h_2 $ & $ h_3 $ & $ h_4 $ & $ \delta $ & $ \gamma $ & $ \gamma_1 $ & $ \gamma_3 $ &  $ \cos\beta $  \\ \midrule
		90                                   & 1,57           & 0,5     & 1,3     & 1,4     & 2,2     & 0,38       & 0,22       & 0,90         & 0,25         & -0,00    \\
		80                                   & 1,40           & 0,4     & 1,3     & 1,4     & 2       & 0,31       & 0,18       & 0,85         & 0,21         & 0,17   \\
		70                                   & 1,22           & 0,4     & 1,3     & 1,3     & 2,2     & 0,31       & 0,26       & 0,85         & 0,30         & 0,34   \\
		60                                   & 1,05           & 0,6     & 1,2     & 1,1     & 2,8     & 0,50       & 0,44       & 0,94         & 0,46         & 0,50   \\
		50                                   & 0,87           & 1,2     & 1,3     & 1,1     & 4,1     & 0,92       & 0,58       & 1,00         & 0,58         & 0,64     \\
		40                                   & 0,70           & 1,6     & 1,3     & 1       & 4,7     & 1,23       & 0,65       & 0,99         & 0,65         & 0,77    \\
		30                                   & 0,52           & 2,9     & 1,3     & 1,5     & 6,8     & 2,23       & 0,64       & 0,92         & 0,69         & 0,87   \\
		20                                   & 0,35           & 3,2     & 1,1     & 1,6     & 7       & 2,91       & 0,63       & 0,87         & 0,72         & 0,94    \\
		10                                   & 0,17           & 2,7     & 1,1     & 1,3     & 6,4     & 2,45       & 0,66       & 0,91         & 0,73         & 0,98    \\
		0                                    & 0,00           & 2,3     & 1,1     & 1,1     & 5,7     & 2,09       & 0,68       & 0,94         & 0,72         & 1,00     \\ \bottomrule
	\end{tabular}
	\caption{Измерения в зависимости от угла $ \beta $}
	\label{table1}
\end{table}

\item Построим график $ \gamma_3(|\cos\beta|) $ и сравним его с теоретической зависимостью \eqref{gamma3}.

  
\begin{figure}[h!]\centering
	\vspace{-1ex}
	\begin{tikzpicture}
	\begin{axis} [ 
		scale=1.5, 
		xlabel={$|\cos\beta|$}, 
		ylabel={$\gamma_3$},
		domain = 0:5, 
		xmin=0,ymin=0, 
		xmax = 1.2, ymax = 1,
		grid = major
	]
	
	\addplot [
		color=black, 
		only marks, 
		error bars/.cd, 
		y dir=both, 
		y explicit
	]
	table [
		col sep=semicolon, 
		x index=1, 
		y index=0
	] 
	{data1.csv};
	
	
	\addplot[color=darkgray] {0.5761*x+0.1735};
	
	\end{axis}
\end{tikzpicture}
	\caption{График зависимости $ \gamma_3(|\cos\beta|) $}
\end{figure}

\item Исследуем зависимость видности от разности хода между лучами. Для этого установите поляроид П1 в положение, в котором интерференционная картина видна наиболее чётко 
$ (\alpha=0^\circ, \gamma_3 = 1) $. Результаты измерений занесем
в таблицу \ref{table2}.

\begin{table}[!h]\centering
	\begin{tabular}{c|cccc|cccc} \toprule
$ x $,~см & $ h_1 $ & $ h_2 $ & $ h_3 $ & $ h_4 $ & $ \delta $ & $ \gamma $ & $ \gamma_1 $ & $ \gamma_2 $ \\ \midrule
9              & 2,4     & 0,8     & 1,4     & 4,6     & 0,33       & 0,53       & 0,87         & 0,62         \\
11             & 2,2     & 2       & 1,2     & 7       & 0,91       & 0,71       & 1,00         & 0,71         \\
13             & 2,2     & 1,7     & 1       & 6,5     & 0,77       & 0,73       & 0,99         & 0,74         \\
15             & 2,1     & 1,6     & 0,8     & 6,6     & 0,76       & 0,78       & 0,99         & 0,79         \\
18             & 2,1     & 0,8     & 1       & 4,8     & 0,38       & 0,66       & 0,89         & 0,73         \\
20             & 2,1     & 0,5     & 1,2     & 4       & 0,24       & 0,54       & 0,79         & 0,68         \\
22             & 2,2     & 2,8     & 2       & 8       & 1,27       & 0,60       & 0,99         & 0,60         \\
24             & 2,2     & 1       & 1,8     & 4,5     & 0,45       & 0,43       & 0,93         & 0,46         \\
26             & 2,6     & 1,6     & 2,8     & 5,6     & 0,62       & 0,33       & 0,97         & 0,34         \\
28             & 2,4     & 0       & 2,4     & 2,6     & 0,00       & 0,04       & 1,00         & 0,04         \\
34             & 2,5     & 3,2     & 5,5     & 5,8     & 1,28       & 0,03       & 0,99         & 0,03         \\
40             & 2,8     & 0       & 2,7     & 2,8     & 0,00       & 0,02       & 1,00         & 0,02         \\
44             & 2,8     & 2,5     & 4,6     & 5,8     & 0,89       & 0,12       & 1,00         & 0,12         \\
50             & 3       & 0,6     & 3,2     & 3,8     & 0,20       & 0,09       & 0,75         & 0,11         \\
54             & 2,8     & 0       & 2,7     & 3       & 0,00       & 0,05       & 1,00         & 0,05         \\
58             & 2,6     & 1,2     & 4       & 4,7     & 0,46       & 0,08       & 0,93         & 0,09         \\
60             & 3,2     & 2       & 4,8     & 5,7     & 0,63       & 0,09       & 0,97         & 0,09         \\
64             & 2,6     & 3,8     & 6       & 6,5     & 1,46       & 0,04       & 0,98         & 0,04         \\
66             & 2,6     & 3,6     & 5,5     & 6,8     & 1,38       & 0,11       & 0,99         & 0,11         \\
68             & 2,6     & 2,6     & 4,2     & 6,2     & 1,00       & 0,19       & 1,00         & 0,19         \\
70             & 2       & 2       & 2,8     & 4,6     & 1,00       & 0,24       & 1,00         & 0,24         \\
72             & 2       & 1,5     & 2       & 4,5     & 0,75       & 0,38       & 0,99         & 0,39         \\
74             & 2       & 1,2     & 1,7     & 4,4     & 0,60       & 0,44       & 0,97         & 0,46         \\
76             & 2       & 1,6     & 1,5     & 5,5     & 0,80       & 0,57       & 0,99         & 0,57         \\
78             & 2       & 1,6     & 1       & 6       & 0,80       & 0,71       & 0,99         & 0,72         \\
80             & 2       & 0,7     & 0,9     & 4,2     & 0,35       & 0,65       & 0,88         & 0,74         \\
82             & 2       & 1,4     & 1       & 5,5     & 0,70       & 0,69       & 0,98         & 0,70         \\
84             & 2       & 0,5     & 1,1     & 3,7     & 0,25       & 0,54       & 0,80         & 0,68         \\
86             & 2       & 1,6     & 1,5     & 5,4     & 0,80       & 0,57       & 0,99         & 0,57         \\
88             & 2       & 0,8     & 1,3     & 4,9     & 0,40       & 0,58       & 0,90         & 0,64         \\ \bottomrule
\end{tabular}
\caption{Измерения в зависимости от разности хода}
\label{table2}
\end{table}

Построим график зависимости видности $ \gamma_2(x)  $от координаты блока Б2.
\newpage

\begin{figure}[h!]\centering
	\begin{tikzpicture}
	\begin{axis} [ 
		scale=1.7, 
		xlabel={$x$}, 
		ylabel={$\gamma_2$},
		domain = 0:5, 
		xmin=0,ymin=0,
		grid = major
	]
	
	\addplot [
		color=black, 
		only marks
	]
	table [
		col sep=semicolon, 
		x index=1, 
		y index=0
	] {data2.csv};
	
	\end{axis}


\end{tikzpicture}
	\caption{ График зависимости видности $ \gamma_2(x)  $}
\end{figure}

\end{enumerate}
\section{Обработка}
По полученному графику определим примерный размер резонатора лазера: 
\[
	L = 80~см- 15~см = 65~см 
\]

Тогда межмодовое расстояние равно: 
\[
	\Delta\nu = \dfrac{3 \cdot 10^8~м/с}{2\cdot 0,65~м} = 2,3 \cdot 10^8~Гц
\]


Полуширина первого максимума на половине высоты:
\[
	l_{1/2} = 25~см - 15~см = 10~см
\]


Диапазон частот, в котором происходит генерация продольных мод:
\[
	\Delta F = \dfrac{c\sqrt{\ln2}}{\pi l_{1/2}} = \dfrac{3\cdot 10^8~м/с\cdot0,83}{3,1415\cdot0,1~м} = 8\cdot10^8~Гц
\]


Оценим число генерируемых лазером продольных мод:
\[
	N\approx 1 + \dfrac{2\Delta F}{\Delta\nu} = 1 + \dfrac{2\cdot 8}{2,3} = 5
\]
	

\section{Вывод}
Исследуя видность интерференционной картины излучения гелий-неонового лазера мы измерили
диапазон частот, в котором происходит генерация продольных мод. Точно определили размер резонатора. Зависимость $ \nu_3(|\cos\beta|) $ оказалось линейной, но не проходит
через ноль из-за неточности установки и измерений (поляроид не перекрывал свет полностью).	


\end{document}
























